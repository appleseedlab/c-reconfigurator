\section{Syntax}

\subsection{V-AST Syntax}

An abstract syntax tree with variability (V-AST) can be represented using the syntax given in (\ref{eq:vastsyntax}). A tree vertex (or an object) \textit{obj} can be either a presence condition (\textit{pc}), a language element (\textit{lang}) or a generic node (\textit{node}). All V-AST types have corresponding types in the Java/XTend code as shown in Tbl.\,\ref{table:vertmap}.

\begin{equation}
\label{eq:vastsyntax}
\begin{array}{rcl}
obj & ::= & pc ~\mid~ lang ~\mid~ node
\\[0.5mm]
node & ::= & (name,pair)
\\[0.5mm]
pair & ::= & \epsilon ~\mid~ obj::pair
\end{array}
\end{equation}

\paragraph{Objects.} Objects are represented only as identifiers \obj{o_1}, \obj{o_2}.

\paragraph{Nodes.} A \type{node} is a tuple of two elements: a string \type{name} and a \type{pair} which contains the children. It can be represented as an identifier \node{n_1},  \node{n_2}, as a tuple (\name{Conditional},\pair{p}) or by using the \type{name} \node{\name{Conditional}}. The \type{name} of the \type{node} can be accessed through the function \func{name} (e.g. \node{\name{Conditional}}.\func{name}). The \type{node} type also has an index accessor for its children (e.g. (\name{Conditional},\pair{p})[i] is accessing child i as ordered in the \type{pair} \pair{p}).

\paragraph{Pair.}  A \textit{pair} can either be empty ($\epsilon$) or it can be formed of a head of type \textit{obj} and a tail of type \textit{pair}. The head and tail can be accessed through their respective functions: \pair{\id{p_1}}.\func{head} and \pair{\id{p_1}}.\func{tail}.

\begin{table}[h]
\begin{minipage}[t]{0.4\textwidth}
\centering
\begin{tabular}{| r | l |}
	\hline
	\textbf{V-AST} & \textbf{Java/XTend} \\\hline
	\type{obj}  & Object \\\hline
	\type{pc}   & PresenceCondition \\\hline
	\type{lang} & Language<CTag> \\\hline
	\type{node} & GNode \\\hline
	\type{pair} & Pair<Object> \\\hline
\end{tabular}
\caption{Mapping from V-AST vertices to Java classes.}
\label{table:vertmap}
\end{minipage}
\hspace*{-\textwidth} \hfill
\begin{minipage}[t]{0.55\textwidth}
\centering
\begin{tabular}{| r | p{5.1cm} |}
	\hline
	\textbf{type} & \textbf{representations} \\\hline
	\type{obj}  & \obj{o_1}, \obj{o_2} \\\hline
	\type{pc}   & \pc{pc_1}, \pc{pc_2}, \id{\phi_1}, \id{\psi_1}, \pc{\name{true}}, \pc{\name{false}}, $\phi \land \name{false}$ \\\hline
	\type{lang} & \lang{ln} \\\hline
	\type{node} & \node{n}, (\name{Conditional},\pair{p}), \node{\name{Conditional}} \\\hline
	\type{pair} & \pair{p_0} \\\hline
\end{tabular}
\caption{Various representations for vertices.}
\label{table:representations}
\end{minipage}
\end{table}






\subsection{Rule Syntax}

A rule has a name, an input pattern, an output patten and an algorithm.

%--- RemOneRule ---%
\noindent
\begin{tabular}{| p{.2\textwidth} | p{.2\textwidth} | p{.6\textwidth} |}
	\hline
	\multicolumn{3}{|l|}{\textsc{RemOneRule}} \\\hline
	Input \pair{\id{pair}} & Output \pair{\_} & Algorithm \\\hline

\begin{vastcode}
/ @\node{\name{Conditional}}@
| - @\pc{\name{true}}@
| = @\pair{\id{children}}@
= @\pair{\id{tail}}@
\end{vastcode} &

\begin{vastcode}
/ @\pair{\id{children}}@
= @\pair{\id{tail}}@
\end{vastcode} &

\begin{PseudoCode}
preconditions:
  @\pair{\id{pair}} != $\epsilon$@
  @\pair{\id{pair}}.\func{head}@ : (@\name{Conditional}@,@\_@)
  @\pair{\id{pair}}.\func{head}@.filter(@\type{pc}@).size = 1
  @\pair{\id{pair}}.\func{head}@[0] = @\pc{\name{true}}@

do:
  return @\pair{\id{pair}}.\func{head}@
    @.\func{getChildrenGuardedBy}(\pc{\name{true}})@
    @.\func{append}(\pair{\id{pair}}.\func{tail})@
\end{PseudoCode} \\\hline
\end{tabular}




%--- RemZeroRule ---%
\noindent
\begin{tabular}{| p{.2\textwidth} | p{.2\textwidth} | p{.6\textwidth} |}
	\hline
	\multicolumn{3}{|l|}{\textsc{RemZeroRule}} \\\hline
	Input \pair{\id{pair}} & Output \pair{\_} & Algorithm \\\hline
	
\begin{vastcode}
/ @\node{\name{Conditional}}@
| - @\pc{\name{false}}@
| = @\pair{\id{children}}@
= @\pair{\id{tail}}@
\end{vastcode} &

\begin{vastcode}
@\pair{\id{tail}}@
\end{vastcode} &
	
\begin{PseudoCode}
preconditions:
  @\pair{\id{pair}} != $\epsilon$@
  @\pair{\id{pair}}.\func{head}@ : (@\name{Conditional}@,@\_@)
  @\pair{\id{pair}}.\func{head}@.filter(@\type{pc}@).size = 1
  @\pair{\id{pair}}.\func{head}@[0] = @\pc{\name{false}}@
	
do:
  return @\pair{\id{pair}}.\func{tail}@
\end{PseudoCode} \\\hline
\end{tabular}




%--- SplitConditionalRule ---%
\noindent
\begin{tabular}{| p{.2\textwidth} | p{.2\textwidth} | p{.6\textwidth} |}
\hline
\multicolumn{3}{|l|}{\textsc{SplitConditionalRule}} \\\hline
Input \pair{\id{pair}} & Output \pair{\_} & Algorithm \\\hline

\begin{vastcode}
/ @\node{\name{Conditional}}@
| - @\id{\phi_1}@
| - @\pair{\id{children_1}}@
| - @\id{\phi_2}@
| - @\pair{\id{children_2}}@
| - @\id{\phi_n}@
| = @\pair{\id{children_n}}@
= @\pair{\id{tail}}@
\end{vastcode} &

\begin{vastcode}
/ @\node{\name{Conditional}}@
| - @\id{\phi_1}@
| = @\pair{\id{children_1}}@
- @\node{\name{Conditional}}@
| - @\id{\phi_2}@
| = @\pair{\id{children_2}}@
- @\node{\name{Conditional}}@
| - @\id{\phi_n}@
| = @\pair{\id{children_n}}@
= @\pair{\id{tail}}@
\end{vastcode} &

\begin{PseudoCode}
preconditions:
  @\pair{\id{pair}} != $\epsilon$@
  @\pair{\id{pair}}.\func{head}@ : (@\name{Conditional}@,@\_@)
  @\pair{\id{pair}}.\func{head}@.filter(@\type{pc}@).size >= 2

do:
  @\pair{\id{newPair}} = $\epsilon$@

  for @(\id{\phi_i} in [\id{\phi_1}..\id{\phi_n}])@
    @\pair{\id{newPair}} = \pair{\id{newPair}}.\func{append}(@
      @(\name{Conditional},@
      @\id{\phi_i}::\pair{\id{pair}}.\func{head}.\func{getChildrenGuardedBy}(\id{\phi_i})))@

  return @\pair{\id{pair}}.\func{tail}@
\end{PseudoCode} \\\hline
\end{tabular}




%--- ConstrainNestedConditionalsRule ---%
\noindent
\begin{tabular}{| p{.2\textwidth} | p{.3\textwidth} | p{.5\textwidth} |}
\hline
\multicolumn{3}{|l|}{\textsc{ConstrainNestedConditionalsRule}} \\\hline
Input \node{\id{node}} & Output \node{\_} & Algorithm \\\hline

\begin{vastcode}
ancestors:
@(\name{Conditional},\id{\psi_1}::\pair{\_})@
@(\name{Conditional},\id{\psi_2}::\pair{\_})@
@(\name{Conditional},\id{\psi_n}::\pair{\_})@

@\node{\name{Conditional}}@
- @\id{\phi_1}@
= @\pair{\id{children}}@
\end{vastcode} &

\begin{vastcode}
@\node{\name{Conditional}}@
- @\func{constrain}(\id{\phi_1}, $\id{\psi_1} \land \id{\psi_2} \land \id{\psi_n}$)@
= @\pair{\id{children}}@
\end{vastcode} &

\begin{PseudoCode}
preconditions:
@\node{\id{node}}.\func{name} = \name{Conditional}@
@\node{\id{node}}@.filter(@\type{pc}@).size = 1
	
do:
@\pc{\id{simpl}} = \func{constrain}(\id{\phi}, \node{\id{node}}.\func{presenceCondition})@

if @(\pc{\id{simpl}} != \id{\phi})@
  return @(\name{Conditional},\pc{\id{simpl}}::\node{\id{node}}.\func{toPair}.\func{tail})@
else
  return @\node{\id{node}}@
\end{PseudoCode} \\\hline
\end{tabular}




%--- ConditionPushDownRule ---%
\noindent
\begin{tabular}{| p{.25\textwidth} | p{.2\textwidth} | p{.55\textwidth} |}
\hline
\multicolumn{3}{|l|}{\textsc{ConditionPushDownRule}} \\\hline
Input \pair{\id{pair}} & Output \pair{\_} & Algorithm \\\hline

\begin{vastcode}
/ @\node{\name{Conditional}}@
| - @\id{\psi_1}@
| - @\node{\name{Conditional}}@
| | - @\id{\phi_{11}}@
| | = @\pair{\id{children_{11}}}@
| - @\id{\psi_2}@
| - @\node{\name{Conditional}}@
| | - @\id{\phi_{21}}@
| | - @\pair{\id{children_{21}}}@
| | - @\id{\phi_{22}}@
| | = @\pair{\id{children_{22}}}@
| - @\node{\name{Conditional}}@
| | - @\id{\phi_{31}}@
| | = @\pair{\id{children_{31}}}@
| - @\id{\psi_n}@
| = @\node{\name{Conditional}}@
|   - @\id{\phi_{n1}}@
|   = @\pair{\id{children_{n1}}}@
= @\pair{\id{tail}}@
\end{vastcode} &

\begin{vastcode}
/ @\node{\name{Conditional}}@
| - @$\id{\phi_{11}} \land \id{\psi_1}$@
| = @\pair{\id{children_{11}}}@
- @\node{\name{Conditional}}@
| - @$\id{\phi_{21}} \land \id{\psi_2}$@
| - @\pair{\id{children_{21}}}@
| - @$\id{\phi_{22}} \land \id{\psi_2}$@
| = @\pair{\id{children_{22}}}@
- @\node{\name{Conditional}}@
| - @$\id{\phi_{31}} \land \id{\psi_2}$@
| = @\pair{\id{children_{31}}}@
- @\node{\name{Conditional}}@
| - @$\id{\phi_{n1}} \land \id{\psi_n}$@
| = @\pair{\id{children_{n1}}}@
= @\pair{\id{tail}}@
\end{vastcode} &

\begin{PseudoCode}
preconditions:
@\pair{\id{pair}} != $\epsilon$@
@\pair{\id{pair}}.\func{head}@ : @(\name{Conditional},\_)@
@\pair{\id{pair}}.\func{head}@.forall[it:@\pc{\_} $\lor$ @it:@(\name{Conditional},\_)@]

do:
@\pair{\id{pair}}.\func{head}@.filter(@\type{cond}@).map[@\node{node}@ | 
  (@\name{Conditional},\node{node}@.map[@\obj{child}@ |
    if (@\obj{child}@ : @\pc{\_}@) @\pc{child} $\land$ \func{pcOf}(\node{node})@
    else @\obj{child}@
  ])
].@\func{append}(\pair{\id{tail}})@
\end{PseudoCode} \\\hline
\end{tabular}




%--- MergeSequentialMutexConditionalRule ---%
\noindent
\begin{tabular}{| p{.2\textwidth} | p{.2\textwidth} | p{.6\textwidth} |}
\hline
\multicolumn{3}{|l|}{\textsc{MergeSequentialMutexConditionalRule}} \\\hline
Input \pair{\id{pair}} & Output \pair{\_} & Algorithm \\\hline

\begin{vastcode}
/ @\node{\name{Conditional}}@
| - @\id{\phi_1}@
| = @\pair{\id{children_1}}@
- @\node{\name{Conditional}}@
| - @\id{\phi_2}@
| = @\pair{\id{children_2}}@
= @\pair{\id{tail}}@
\end{vastcode} &

\begin{vastcode}
/ @\node{\name{Conditional}}@
| - @\id{\phi_1} $\lor$ \id{\phi_2}@
| = @\pair{\id{children_1}}@
= @\pair{\id{tail}}@
\end{vastcode} &

\begin{PseudoCode}
preconditions:
  @\pair{\id{pair}} != $\epsilon$@
  @\pair{\id{pair}}.\func{size} >= 2@
  @\pair{\id{pair}}.\func{head}@ : @(\name{Conditional},\_)@
  @\pair{\id{pair}}.\func{head}@.filter(@\type{cond}@).@\func{size}@ == 1
  @\pair{\id{pair}}.\func{tail}.\func{head}@ : @(\name{Conditional},\_)@
  @\pair{\id{pair}}.\func{tail}.\func{head}@.filter(@\type{cond}@).@\func{size}@ == 1
  @\func{areMutex}(\id{\phi_1},\id{\phi_2})@
  @\func{structurallyEquals}(\pair{\id{children_1}},\pair{\id{children_2}})@

do:
  @(\name{Conditional}, \id{\phi_1} $\lor$ \id{\phi_2} :: \pair{\id{children_1}}) :: \pair{\id{tail}}@
\end{PseudoCode} \\\hline
\end{tabular}






\subsection{Functions}

% http://gauss.ececs.uc.edu/sbsat_user_manual/node77.html
\paragraph{Constrain / Generalized cofactor.} The \func{constrain} function (\ref{eq:constrain}), a.k.a. the generalized cofactor, constrains a presence condition \id{\phi} by another presence condition \id{\psi} that has already been decided to be true and eliminates any redundant variables.

\begin{equation}
\label{eq:constrain}
\begin{array}{l}
\func{models}(\phi) = \{ M \!\mid\! M \implies \phi \}
\\[0.5mm]
\func{constrain}(\phi,\psi) = \phi' \bigg|
	\begin{array}{lr}
	\phi' \!\land\! \psi \implies \phi & and \\
	\func{models}(\phi') = \func{models}(\phi) \backslash \func{models}(\psi) & 
	\end{array}
\end{array}
\end{equation}

For example, $\func{constrain}(A \land C, A \land B)=C$ shows that, having decided that $A \land B$ holds, $C$ is the minimal proposition required to show that $A \land C$ also holds: $C \land A \land B \to A \land C$.

Another example: $\func{constrain}(B \lor C, A \land B)=\name{true}$ shows that, having decided that $A \land B$ holds, \name{true} is the minimal proposition required to show that $B \lor C$ also holds: $\name{true} \land A \land B \to B \lor C$.
